\subsection{نتیجه‌گیری}
  از نمودارها و جداول بالا چندین نتیجه‌ می‌گیریم
 \begin{enumerate}
     \item می‌توان مشاهده کرد که در تست‌هایی که در ماشین حقیقی گرفته‌شده‌اند، با رفتن به سمت write بیشتر، ابتدا تعداد remote shootdown 
     ها کم شده و به  
     local mm shootdown 
     ها اضافه شده اما وقتی تعداد write 
     ها خیلی زیاد شده دوباره remote shootdown ها 
     زیاد شده‌اند.
    \item احتمالا تعداد زیاد remote ipi send 
    ها به دلیل این است که این نوع دیتابیس‌ها بیشترین میزان موازی‌سازی را دارند و همچنین طبق گفته‌های ردیس یکی از عوامل اصلی آن فراخوانی 
    madvise 
    است که مشاهده می‌کنیم در این تست‌ها به میزان خیلی زیادی وجود دارد.
    \item در ماشین حقیقی هر چه به سمت نوشتن بیشتر می‌رویم، تعداد فراخوانی‌های سیستمی  mmap و madvise و msync کمتر می‌شود و اما تعداد TLB Shootdown ها ابتدا کم و سپس زیاد می‌شود که یعنی احتمالا یک فراخوانی دیگر منجر به این تغییرات شده است.
    \item طبق تحلیل می‌توان دید تنها فراخوانی‌هایی که از این تغییرات پیروی می‌کنند، فراخوانی‌های 
    mprotect و 
    munmap 
    هستند که یعنی احتمالا تغییر این دو بسیار در tlb shootdown
    های ما تعیین‌کننده‌تر از باقی عملیات‌ها است.
    \item در ماشین حقیقی بهترین حالت از نظر تعداد shootdown ها و miss ها حالت 
    ۵ - ۵ 
    است که همان نصف خواندن نصف نوشتن است که تقریبا بین این سه حالت و تعداد فراخوانی‌ها حالت مینیمم نسبی را دارد.
    \item 
    در مقایسه ماشین مجازی و حقیقی متوجه می‌شویم که در ماشین مجازی، تعداد miss ها و نیز tlb shootdown 
    ها به مقدار قابل‌توجهی کمتر از ماشین حقیقی است که نشان می‌دهد در ماشین مجازی در این سیستم توزیع شده عملکرد بسیار بهتری به نسبت ماشین حقیقی داریم.
    \item 
    با تقریب خوبی،‌ 
    futex 
    بیشترین فراخوانی سیستمی است که با توجه به اینکه این سیستم حالت توزیع‌شده دارد و نیز نیاز به data consistancy زیاد بین node ها دارد،‌ احتمالا از این فراخوان برای synchronization استفاده می‌شود و طی تحقیقات انجام شده این فراخوانی سیستمی می‌تواند در tlb shootdown در 
    حالت‌های خاص تاثیر بگذارد که امکان دارد این یکی از آن حالت‌های خاص باشد با توجه به اینکه در تست 5 - 5 ،
    این فراخوانی سیستمی از بقیه کمتر تکرار شده.
    \item 
    در ماشین مجازی تقریبا نسبت‌ علت‌های shootdown 
    های مختلف ثابت مانده است که نشان می‌دهد که در ماشین مجازی عملکرد قابل پیش‌بینی‌تری به نسبت ماشین حقیقی دارد که این یک نکته مثبت در استفاده از ماشین مجازی است.
    \item 
    یکی از فراخوانی‌های سیستمی که در ماشین مجازی مشاهده می‌شود که در ماشین حقیقی نیست 
    sched\_yield
    است که می‌توان دید با remote ipi send 
    رابطه معکوس دارد. البته این فراخوانی سیستمی عملا باعث می‌شود که پردازه به آخر صف اجرا برود.
    طبق تحقیقات به عمل آمده،‌ این فراخوانی در ماشین‌های مجازی می‌تواند منجر به توقف یک vCPU برای مدت زمان طولانی بشود که اصلا اتفاق بهینه‌ای نیست.
    \item ترتیب تغییر msync نیز همانند تغییر sched\_yield و نیز مطابق بر تغییرات کلی remote shootdown است که شاید در ارزیابی نهایی ما تاثیر داشته باشد.
 \end{enumerate}