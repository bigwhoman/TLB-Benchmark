
\subsection{مقدمه}
\subsubsection{نصب Cassandra}
کاساندرا یک دیتابیس توزیع‌شده nosql است.\\
برای نصب و استفاده از آن از داکر کمک می‌گیریم.
ابتدا برای راه‌اندازی باید فایل 
docker compose 
زیر را راه‌اندازی کنیم. می‌دانیم در صورتی که 
image  
خواسته شده، موجود نباشد، خود داکر کامپوز آنرا
pull 
می‌کند.\\
دلیل استفاده از داکر این بود که می‌توانستیم نحوه کارکرد 
seed 
های 
cassandra 
را در بین 
node 
های دنیای واقعی که همان کامپیوترهای مختلف هستند بررسی کنیم.


\begin{latin}
  \begin{verbatim}
    version: '3'
    services:
      cassandra_node_template:
        image: cassandra
        environment:
          - CASSANDRA_CLUSTER_NAME=my_cluster
          - CASSANDRA_ENDPOINT_SNITCH=GossipingPropertyFileSnitch
        networks:
          - cassandra_net
    networks:
      cassandra_net:
        driver: bridge
  \end{verbatim}
\end{latin}

حال این ایمیج برای ساختن یک عدد 
container 
استفاده می‌شود.
\\
برای ساخت کلیه موارد مورد نظر، از اسکریپت create\_nodes استفاده می‌کنیم که عملا تعداد seed های مورد نظر را 
گرفته و به تعداد آنان، کانتینر حاوی 
cassandra 
ایجاد کرده، 
این کانتینرها را به هم متصل می‌کند و یک 
cluster 
تشکیل می‌دهد و درنهایت 
صبر می‌کند تا آنها بالا بیایند و از صحت آنان مطلع می‌شود.
\\
\textbf{نکته مهم : } 
به علت عدم وجود حافظه کافی و اینکه هر کدام از 
seed 
های کاساندرا مقدار زیادی منبع مصرف می‌کنند، نتوانستم بیش از ۳ 
node 
همزمان را بالا بیاورم پس مجبور شدم برای جلوگیری از کاهش عملکرد، صرفا به ۲ seed 
اکتفا کنم.
\\
حال پس از بالا آمدن node 
ها باید آنها را بنچ‌مارک کنیم که برای این کار از خود ابزار 
\textbf{cassandra-stress}
موجود در tools 
همراه برنامه استفاده می‌کنیم.
\\
متاسفانه این برنامه در کانتینرها و کلا docker image 
خود کاساندرا موجود نبود و مجبور شدم به صورت جداگانه کاساندرا را در کامپیوتر نصب کنم تا بتوانم از این ابزار استفاده کنم.\\
حال از اسکریپت benchmark.sh 
برای بنچ‌مارک با کمک کاساندرا استفاده می‌کنیم.
برای این کار اسکریپت زیر را درنظر بگیرید.
\codebox{

sudo perf record -o ./benchmark/cassandra-benchmark-3-7-bare-metal-\$(uname -r).perf -e tlb:tlb\_flush,dTLB-loads,dTLB-load-misses,iTLB-load-misses,cache-misses,page-faults cassandra-stress mixed ratio(write=3,read=7) n=1000000 -rate threads=50
\\
strace cassandra-stress mixed ratio(write=3,read=7) n=1000000 -rate threads=50 2> ./benchmark/cassandra-benchmark-3-7-bare-metal-\$(uname -r).strace 
\\

sudo perf record -o ./benchmark/cassandra-benchmark-5-5-bare-metal-\$(uname -r).perf -e tlb:tlb\_flush,dTLB-loads,dTLB-load-misses,iTLB-load-misses,cache-misses,page-faults cassandra-stress mixed ratio(write=3,read=7) n=1000000 -rate threads=50
\\
strace cassandra-stress mixed ratio(write=5,read=5) n=1000000 -rate threads=50 2> ./benchmark/cassandra-benchmark-5-5-bare-metal-\$(uname -r).strace 
\\

sudo perf record -o ./benchmark/cassandra-benchmark-7-3-bare-metal-\$(uname -r).perf -e tlb:tlb\_flush,dTLB-loads,dTLB-load-misses,iTLB-load-misses,cache-misses,page-faults cassandra-stress mixed ratio(write=3,read=7) n=1000000 -rate threads=50
\\
strace cassandra-stress mixed ratio(write=7,read=3) n=1000000 -rate threads=50 2> ./benchmark/cassandra-benchmark-7-3-bare-metal-\$(uname -r).strace 
}
این اسکریپت عملا در ۳ سناریو، یک میلیون تا عملیات را برروی 
seed 
های کاساندرا انجام می‌دهد.
\\
گزینه mixed مشخص می‌کند که این 
نسبت خواندن و نوشتن‌ها به چه اندازه باشد و نیز 
تعداد عملیات‌های موازی با 
threads 
مشخص شده است.
\\
این ۳ سناریو محبوب‌ترین سناریوهایی بودند که طبق گفته خود 
کامیونیتی کاساندرا 
غالبا در سیستم‌های توزیع‌شده استفاده می‌شوند.
\\
حال نتایج را در سیستم‌های مختلف بررسی می‌کنیم.
