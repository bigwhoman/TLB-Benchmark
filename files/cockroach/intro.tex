\subsection{مقدمه}
\subsubsection{نصب Cockroach}
برای نصب 
redis 
در این قسمت از داکر
استفاده می‌کنیم.
می‌دانیم که داکر 
عملا فقط عملیات 
containrize 
کردن را انجام
می‌دهد
پس  در کلیت تست ما 
تاثیری ندارد.
\\
ابتدا داکر و داکر کامپوز را مطابق مراحل نصب گفته شده در
داکیومنتیشن نصب می‌کنیم و سپس 
برای نصب ردیس، از قطعه کد زیر استفاده می‌کنیم که 
ردیس را نصب می‌کند و پورت ۶۳۷۹ آنرا، به پورت ۶۳۷۹ کامپیوتر بایند می‌کند.
\codebox{
    TODO\\\\
version: '3.7'\\
services:\\
  redis:\\
    image: redis\\
    ports:\\
      - 6379:6379
}
برای بنچ‌مارک ردیس از ابزار درونی خود ردیس استفاده می‌کنیم که همان 
\textbf{redis-benchmark}
است.
\\
این ابزار در چندین سناریوی ازپیش‌ تعریف‌‌شده دیتابیس گفته‌شده را تست می‌کند و همچنین می‌توان تعداد کاربران و نیز
تراکنش‌های موازی آنرا تعیین کرد.
\\
برای تست کردن و جمع‌آوری دیتا از اسکریپت زیر استفاده می‌کنیم.
\begin{latin}
  \codebox{
  sudo perf record -o ./benchmark/redis-benchmark-bare-metal-\$(uname -r).perf -e tlb:tlb\_flush,dTLB-loads,dTLB-load-misses,iTLB-load-misses,cache-misses,page-faults redis-benchmark
  \\
  strace redis-benchmark 2> ./benchmark/redis-benchmark-bare-metal-\$(uname -r).strace 
}
\end{latin}
\noindent
برای اینکه بدانیم هر تست برای کدام ورژن کرنل است در انتهای نام تست، نام کرنل را هم می‌گذاریم تا بعدا برای 
مقایسه آنها فرایند ساده‌تری داشته باشیم.

