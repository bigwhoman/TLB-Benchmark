\subsection{مقدمه}
\subsubsection{نصب Cockroach}
برای استفاده از CockroachDB 
مانند قسمت‌های قبلی، از داکر و داکرکامپوز
استفاده می‌کنیم به این صورت که ابتدا از اسکریپت
create\_cockroach\_nodes
برای ساخت تعداد دلخواه node 
استفاده می‌کنیم که این اسکریپت آنها را در یک شبکه قرار می‌دهد و نیز آنها را به یکدیگر متصل می‌کند.
\\
حال یک نمونه از فایل داکر ساخته شده برای 3 عدد node را می‌توان مشاهده کرد :
\begin{latin}
  \begin{verbatim}
    version: '3'
    services:
      roach1:
        image: cockroachdb/cockroach
        command: start --advertise-addr=roach1:26357 
                  --http-addr=roach1:8080 --listen-addr=roach1:26357 
                  --store=tpcc-local1 --sql-addr=roach1:26257 
                  --insecure --join=roach1:26357,roach2:26357,roach3:26357
        hostname: roach1
        ports:
          - "26257:26257"
          - "8080:8080"
        volumes:
          - ./cockroach-data/roach1:/cockroach/cockroach-data
        networks:
          - crdb_network
      roach2:
        image: cockroachdb/cockroach
        command: start --advertise-addr=roach2:26357
                    --http-addr=roach2:8081 --listen-addr=roach2:26357
                    --store=tpcc-local2 --sql-addr=roach2:26258
                    --insecure --join=roach1:26357,roach2:26357,roach3:26357
        hostname: roach2
        ports:
          - "26258:26258"
          - "8081:8081"
        volumes:
          - ./cockroach-data/roach2:/cockroach/cockroach-data
        networks:
          - crdb_network
      roach3:
        image: cockroachdb/cockroach
        command: start --advertise-addr=roach3:26357
                    --http-addr=roach3:8082 --listen-addr=roach3:26357
                    --store=tpcc-local2 --sql-addr=roach3:26259
                    --insecure --join=roach1:26357,roach2:26357,roach3:26357
        hostname: roach3
        ports:
          - "26259:26259"
          - "8082:8082"
        volumes:
          - ./cockroach-data/roach3:/cockroach/cockroach-data
        networks:
          - crdb_network
    networks:
      crdb_network:
        driver: bridge

  \end{verbatim}
\end{latin}

حال پس از راه‌اندازی node ها اسکریپت benchmark.sh را ران می‌کنیم که 
تست TPC-C 
را اجرا می‌کند.
این تست یکی از معتبرترین و واقعی‌ترین تست‌های دیتابیس‌های توریع‌شده است که 
یک فروشگاه آنلاین را به همراه تعداد انبار داده شده مثلا آمازون و تراکنش‌های آن، شبیه‌سازی می‌کند.
برای اطلاع بیشتر از این تست بسیار جالب به 
\link{https://www.tpc.org/tpc_documents_current_versions/pdf/tpc-c_v5.11.0.pdf}{این لینک}
مراجعه کنید.
\\
طبق عکس زیر مشاهده می‌کنیم که برای تست به صورت لوکال باید از ۳ 
node استفاده کنیم.
\\
\begin{figure}[H]
  \centering
  \includegraphics[scale=0.5]{pictures/cockroach/init/tpcc2.png}
  \caption{تنظیمات \lr{benchmark}}
\end{figure}
حال برای انجام این تست ابتدا باید در یکی از node 
های کلاستر، داده‌های تست را لود کنیم و دیتابیس را بسازیم.
\\
\codebox{
  cockroach workload fixtures import tpcc --warehouses=10 'postgresql://localhost:26257?sslmode=disable'
}
\begin{figure}[H]
  \centering
  \includegraphics[scale=0.25]{pictures/cockroach/init/init-tpcc2.png}
  \caption{نتیجه import داده}
\end{figure}
سپس تست را با دستور زیر شروع می‌کنیم.
این تست به مدت ۱۰ دقیقه اجرا می‌شود.
\codebox{
  sudo perf record -p \$(pidof cockroach | cut -d' ' -f2- | tr ' ' ',') -o ./benchmark/cockroach-benchmark-tpcc-bare-metal-\$(uname -r).perf -e tlb:tlb\_flush,dTLB-loads,dTLB-load-misses,iTLB-load-misses,cache-misses,page-faults
}