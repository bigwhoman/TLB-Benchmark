\subsection{نتیجه‌گیری}
از نمودار و نتایج بالا چند نتیجه می‌گیریم
\begin{enumerate}
    \item اول از همه مشاهده می‌کنیم که به نسبت سیستم توزیع‌شده قبلی تعداد miss ها به شدت بالاتر است که احتمالا به این علت است که cassandra 
    عملا 
    node
    ها را در یک محیط قرار می‌دهد و به صورت خودکار بدون اتصال به شبکه نیز می‌توانند با یکدیگر ارتباط برقرار کنند اما cockroach
    واقعا نیاز به ارتباط بین شبکه‌ای،‌ حتی در صورت وجود 
    node 
    ها در یک سیستم دارد.
    \item
    مانند cassandra 
    در این دیتابیس هم بیشترین فراخوانی سیستمی متعلق به 
    futex 
    است که احتمالا این فراخوانی به علت نیاز فراوان در 
    data consistancy 
    سیستم‌های توزیع شده و وجود node 
    ها با داده‌های یکسان، بسیار به کار می‌رود.
    \item 
    طبق تحقیقات صورت گرفته، فراخوانی fdatasync برای
     سینک کردن دیتا در core به کار می‌رود اما تغییری در مپینگ داده‌ها به وجود نمی‌آورد اما باز هم قابل بررسی است.
     \item 
     تعداد remote ipi send ها در ماشین حقیقی به شدت زیاد است که احتمالا به دلیل وجود تعداد بسیار زیاد فراخوانی madvise در این سیستم است.
     \item 
     تعداد miss ها در ماشین مجازی بسیار بیشتر از ماشین حقیقی است اما تعداد shootdown ها در ماشین مجازی از حقیقی کمتر است که خوب نشان می‌دهد در برخی اوقات تعداد miss ها در ماشین مجازی منجر به shootdown نشده که این در مواقعی می‌تواند خطرناک باشد.
     \item 
     تعداد بالای remote shootdown در ماشین مجازی احتمالا به این علت است که تعداد فراخوانی‌های سیستمی که منجر به آنها می‌شوند مانند mmap, munmap, mprotect 
     بسیار بیشتر از ماشین حقیقی است.
    \item 
    در نهایت می‌توان نتیجه گرفت احتمالا در صورت داشتن یک سیستم توزیع‌شده با دیتابیس cockroach
    ، اسفتاده از یک سیستم حقیقی به نتایج بهتری به نسبت سیستم مجازی منجر می‌شود.
\end{enumerate}