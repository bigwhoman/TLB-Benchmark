\subsection{MySQL همراه با Pages Huge}
در ابتدا باید مانند
\lr{PostgreSQL}
به
\lr{MySQL}
بفهمانیم که باید از
\lr{huge pages}
استفاده بکند.

برای اینکار با توجه به
\link{https://dev.mysql.com/doc/refman/8.0/en/large-page-support.html}{داکیومنشن}
خود
\lr{MySQL}
در فایل کانفیگ
\lr{MySQL}
خط زیر را در زیر
\codeword{[mysqld]}
اضافه کردم.
\codebox{large-pages=ON}
بعد از ری‌استارت سیستم عامل و
\lr{MySQL}
دستور زیر را زدم و نتیجه زیرش آمد:
\codebox{\$ grep $\string^$Huge /proc/meminfo\\
HugePages\_Total:     500\\
HugePages\_Free:      500\\
HugePages\_Rsvd:        0\\
HugePages\_Surp:        0\\
Hugepagesize:       2048 kB\\
Hugetlb:         1024000 kB}
یعنی اصلا
\lr{MySQL} از \lr{huge pages}
استفاده نمی‌کرد. برای همین موضوع کمی در اینترنت سرچ کردم و به
\link{https://stackoverflow.com/q/29457769/4213397}{این}
سوال رسیدم. این سوال نوشته بود که لازم است که دستی خود گروه
\lr{mysql}
را به گروه‌هایی که اجازه‌ی استفاده از
\lr{huge pages}
را داشتند را بدهیم.
\emph{کاری که در PostgreSQL نیازی نبود!}
با این حال به کمک
\codeword{id -g mysql}
شماره گروه
\lr{mysql}
را در آوردم
(که 135 بود)
و خط زیر را به فایل
\codeword{/etc/sysctl.conf}
اضافه کردم.
\codebox{vm.hugetlb\_shm\_group = 135}
سپس کامپیوتر را ری استارت کردم. ولی باز هم مشکل رفع نشد! سپس قطعه کدی که مربوط به فایل
\codeword{/etc/security/limits.conf}
بود را نیز به آن فایل اضافه کردم ولی تاثیری نداشت. سپس شروع به نگاه کردن لاگ‌های
\lr{MySQL}
کردم و خط زیر را در آن دیدم:
\codebox{[Warning] [MY-012681] [InnoDB] large\_page\_aligned\_alloc mmap(138412032 bytes) failed; errno 13}
در ابتدا سعی کردم که بفهمم که منشا این کد کجاست. بعد از صرفا سرچ کردن اسم
\lr{large page aligned alloc}
به
\link{https://dev.mysql.com/doc/dev/mysql-server/latest/large__page__alloc-linux_8h_source.html}{این}
لینک از سورس کد
\lr{MySQL}
رسیدم. در این کد صرفا به کمک تابع
\lr{mmap}
مموری را
\lr{allocate}
می‌کنیم. همچنین با یک سرچ دیگر فهمیدم که
\codeword{errno 13} به \codeword{EACCES}
بر می‌گشت. با توجه به
\lr{man page}ها
می‌توان تعبیر زیر را برای این ارور داشت:
\begin{latin}
\begin{quote}
    A file descriptor refers to a non-regular file.  Or a file  mapping  was  requested,  but  fd  is  not  open  for  reading.  Or MAP\_SHARED was requested and PROT\_WRITE is set, but  fd  is  not open in read/write (O\_RDWR) mode.  Or PROT\_WRITE is set, but the file is append-only.
\end{quote}
\end{latin}
این ارور حقیقتا هیچ کمکی به ما نکرد. من خود تکه کد تابع خود
\lr{MySQL}
را در برنامه‌ی خیلی کوچکی انداختم که صرفا تست کنم آیا کار می‌کند یا خیر و با کمال تعجب کار کرد!
در انتها نیز من به جای یک گیگ
\lr{huge pages}
10 گیگ
\lr{huge pages}
\lr{allocate}
کردم ولی این نیز تاثیری نداشت. پس صرفا از تست کردن
\lr{huge pages}
منصرف شدیم.