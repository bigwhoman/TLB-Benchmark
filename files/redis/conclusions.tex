\subsection{نتیجه‌گیری}
  از نمودارها و جداول بالا چندین نتیجه‌ می‌گیریم
 \begin{enumerate}
     \item تعداد miss های TLB در ماشین مجازی به شدت بیشتر از حقیقی است
     \item توزیع TLB Flush ها در ماشین حقیقی تماما بر local mm shootdown نیست 
     اما در ماشین مجازی به صورت عجیبی تمامی Flush ها انگاری به صورت local انجام می‌شود و تقریبا به هسته‌های دیگر هیچ سینگالی ارسال نمی‌شود
     \item در ماشین مجازی به دلیل عدم ارسال shootdown به هسته‌های دیگر ممکن است در نهایت دچار مشکل در data consistancy شویم که قابل بررسی و درست کردن است.
    \item در هر دو ماشین چندین فراخوانی سیستمی مشاهده می‌کنیم که قابل بررسی‌اند، اولا که فراخوانی‌های سیستمی که احتمالا منجر به TLB Flush می‌شوند، TLB Shootdown 
    می‌شوند طبق تحقیقات من و تمارین درس اصلی‌ترین آنها madvise است که همانطور که مشاهده می‌شود در ماشین حقیقی به تعداد خیلی زیادی از این عملیات وجود دارد که احتمالا عامل اصلی TLB Shootdown ها در ماشین حقیقی و عامل اصلی تفاوت در نمودار این دو ماشین را توجیه می‌کند.
    \item 
    تعداد زیاد remote shutdown ها و remote ipi send ها در ماشین حقیقی به علت این است که طبق
    منابعی که خوانده شده، هنگامی که برنامه multi-threaded داریم، فراخوانی
    madvise
    منجر به تعداد قابل توجهی remote shutdown می‌شود و 
    به این علت که بنچ‌مارک ما عملا ۵۰ ترد موازی هم دارد فراخوانی madvise منجر به 
    shootdown های خیلی زیاد شده.
    \item 
    به صورت کلی نتیجه کلی این است که اجرای کلی برنامه redis در
    ماشین حقیقی به تاخیر کمتری منجر می‌شود و از نظر tlb نیز بهتر است و همچنین احتمالا با کم کردن تعداد madvise ها این عدد به صورت قابل توجهی قابل کاهش باشد.
 \end{enumerate}