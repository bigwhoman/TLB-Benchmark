\subsection{مقدمه}
\subsubsection{نصب Redis}
برای نصب 
redis 
در این قسمت از داکر
استفاده می‌کنیم.
می‌دانیم که داکر 
عملا فقط عملیات 
containrize 
کردن را انجام
می‌دهد
پس  در کلیت تست ما 
تاثیری ندارد.
\\
ابتدا داکر و داکر کامپوز را مطابق مراحل نصب گفته شده در
داکیومنتیشن نصب می‌کنیم و سپس 
برای نصب ردیس، از قطعه کد زیر استفاده می‌کنیم که 
ردیس را نصب می‌کند و پورت ۶۳۷۹ آنرا، به پورت ۶۳۷۹ کامپیوتر بایند می‌کند.
\\
برای بنچ‌مارک از ردیس \textbf{ورژن ۷} که در حال حاضر (۲۰۲۳) 
جدیدترین ورژن است استفاده می‌کنیم.
\begin{latin}
  \begin{verbatim}
    version: '3.7'
    services:
      redis:
        image: redis
        ports:
          - 6379:6379    
  \end{verbatim}
\end{latin}
\noindent
برای بنچ‌مارک ردیس از ابزار درونی خود ردیس استفاده می‌کنیم که همان 
\textbf{redis-benchmark}
است.
\\
این ابزار در 13 سناریوی ازپیش‌ تعریف‌‌شده دیتابیس گفته‌شده را تست می‌کند و همچنین می‌توان تعداد کاربران و نیز
تراکنش‌های موازی آنرا تعیین کرد.
\\
اطلاعات تراکنش‌ها به شرح زیر است :‌
\begin{enumerate}
    \item \lr{100000 Requests}
    \item \lr{50 Parallel Connections}
\end{enumerate}
نکته این است که برای اینکه ابزار بنچ‌مارک تاثیری در پراسس‌های داکر نگذارد و اینکه نتایج دقیق‌تر شوند خود ردیس را به صورت فایل نصبی دانلود می‌کنیم و پردازه داکر را با آن تست می‌کنیم.
برای تست کردن و جمع‌آوری دیتا از اسکریپت زیر استفاده می‌کنیم.
قطعه کد زیر به شماره پردازه redis عملا attach می‌شود و از گرفتن اطلاعات اضافه نیز جلوگیری می‌کند.
\begin{latin}
  \codebox{
    r\_pid=\$(pgrep -f redis)
    \\
    sudo perf record -p \$r\_pid -o ./benchmark/redis-benchmark-bare-metal-\$(uname -r).perf -e tlb:tlb\_flush,dTLB-loads,dTLB-load-misses,iTLB-load-misses,cache-misses,page-faults
}
\codebox{
    strace -f -p "\$r\_pid" 2> ./benchmark/redis-benchmark-bare-metal-\$(uname -r).strace 
}
\end{latin}
\noindent
برای اینکه بدانیم هر تست برای کدام ورژن کرنل است در انتهای نام تست، نام کرنل را هم می‌گذاریم تا بعدا برای 
مقایسه آنها فرایند ساده‌تری داشته باشیم.

